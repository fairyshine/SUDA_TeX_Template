% !Mode:: "TeX:UTF-8"

% 中英文摘要
\begin{cabstract}
	近来出现了许多以带宽换取信噪比的调制方法,比如 PCM 和 PPM,它们的出现进一步激发了人们对广义通信理论的兴趣。在奈奎斯特(Nyquist)和哈特莱(Hartley)发表的一些重要相关论文中,奠定了这一理论的基础。本论文将扩展该理论,增加一些新的因素,具体来说,就是信道中噪声的影响、由于原始消息的统计结构和最终信宿的本质而可能减省的内容。

	通信的基本问题就是在一个地方复现在另一个地方选定的消息,这一复现可能是准确的,也可能是近似的。这些消息通常有特定的含义;也就是说,它们会根据某一系统,与特定的物理或概念实体关联在一起。通信的语义与工程问题无关。重要的是:实际消*是从一个消息集合选出的。所设计的系统必须能够处理任意选定的消息,而不是仅能处理实际选择的特定消息,因为在设计系统时,并不知道会实际选择哪条消息。
	
	如果集合中的消息数目是有限的,而且选择每条消息的可能性相等,那就可以用这个消息数或者它的任意单调函数,来度量从集合中选择一条消息所生成的信息量。正如哈特莱所指出的那样,最自然的选择就是对数函数了。如果考虑消息统计信息的影响,如果消息的选取范围是连续的,那必须对其定义进行重要扩展,但在所有情况下,我们使用的度量在实质上都是对数函数。
	
	——克劳德 E. 香农 1948年 《通信的数学原理》
	\vskip 21bp
	{\heiti\zihao{-4} 关键词:}
	人工智能,
	机器学习,
	物联网,
	边缘计算
	
	\begin{flushright}
		作~~~~~~~~者:XX~~~~XX
		
		指导老师:XXX
		
	\end{flushright}
\end{cabstract}


