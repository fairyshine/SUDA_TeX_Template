\chapter{绪论}

\section{研究背景}

对数度量之所以更为便利,其原因有多种:

1. 它在实践中更为有用。一些在工程上非常重要的参数,比如时间、带宽、延迟数,等等,往往与可能性的数量的对数值呈线性关系。例如,增加一个继电器会使继电器的可能状态数加倍。如果对这一数目求以 2 为底的对数,则增加一个继电器后,会使结果加 1。使时间加倍,会使可能消息数近似变为原来的平方,而其对数则是加倍,诸如此类。

2. 它更接近于人类对正确度量的直观认知。这一点与第 1 个原因密切相关,因为人们在对实体进行直觉度量时,通常是与公共标准进行线性比较。比如,人们认为,两张打孔卡存储信息的容量应当是一张打孔卡的两倍,两个相同信道的信息传输能力应当是一个信道的两倍。

3. 更适于数学运算。许多极限运算很容易用对数表示,如果采用可能性的数目表示,可能会需要进行冗繁、笨拙的重新表述。

\section{研究目的与意义}
\subsection{现有解决方法}

\begin{table}
  \centering
  \begin{tabular}{ccc}
    \toprule
    \textbf{文档域类型} & \textbf{Java类型} & \textbf{宽度(字节)} \\
    \midrule
    BOOLEAN  & boolean &  1 \\
    CHAR     & char    &  2 \\
    BYTE     & byte    &  1 \\
    SHORT    & short   &  2 \\
    INT      & int     &  4 \\
    LONG     & long    &  8 \\
    STRING   & String  &  字符串长度 \\
    DATE     & java.util.Date & 8 \\
    BYTE\_ARRAY & byte$[]$ & 数组长度 \\
    BIG\_INTEGER & java.math.BigInteger & 和具体值有关 \\
    BIG\_DECIMAL & java.math.BigDecimal & 和具体值有关 \\
    \bottomrule
  \end{tabular}
  \caption{测试表格}\label{table:test1}
\end{table}


\subsection{现有问题与不足}

测试一下脚注\footnote{测试脚注},测试一下脚注\footnote{测试脚注},测试一下脚
注\footnote{测试脚注},测试一下脚注\footnote{测试脚注},测试一下脚注\footnote{测
  试脚注},测试一下脚注\footnote{测试脚注},测试一下脚注\footnote{测试脚注},测
试一下脚注\footnote{测试脚注},测试一下脚注\footnote{测试脚注},测试一下脚
注\footnote{测试脚注}。

测试一下引用\cite{ACE05}。

下面是一个项目列表:

\begin{itemize}
\item 这是第一项。这是第一项。
\item 这是第二项。这是第二项。
\item 这是第三项。这是第三项。这是第三项。
  \begin{itemize}
  \item 测试第二层列表。测试第二层列表。
  \item 测试第二层列表。测试第二层列表。
  \begin{itemize}
     \item 测试第三层列表。测试第三层列表。
     \item 测试第三层列表。测试第三层列表。
  \end{itemize}
  \end{itemize}
\end{itemize}

下面是一个编号列表:

\begin{enumerate}
\item 这是第一项。这是第一项。这是第一项。这是第一项。这是第一项。这是第一项。这
  是第一项。这是第一项。这是第一项。这是第一项。这是第一项。
\item 这是第二项。这是第二项。
\item 这是第三项。这是第三项。这是第三项。
  \begin{itemize}
  \item 测试第二层列表。测试第二层列表。
  \item 测试第二层列表。测试第二层列表。
  \item 测试第二层列表。测试第二层列表。测试第二层列表。测试第二层列表。测试第二
    层列表。
  \end{itemize}
\item 这是第四项。这是第四项。这是第四项。
  \item 测试第三层列表。测试第三层列表。测试第三层列表。测试第三层列表。测试第三
    层列表。测试第三层列表。
  \item 测试第三层列表。测试第三层列表。
  \item 测试第三层列表。测试第三层列表。测试第三层列表。
\end{enumerate}

\begin{quote}
这是一段引用。这是一段引用。这是一段引用。这是一段引用。这是一段引用。这是一段引用。
这是一段引用。这是一段引用。这是一段引用。这是一段引用。这是一段引用。这是一段引用。
\end{quote}




\subsection{中心观点与思想}

测试一下定理环境。

\begin{algorithm}[] \caption{一般的蒙特卡洛树搜索方法}\label{algo:raw_MCTS} %MCTS
    
    \SetKw{Func}{Function:}
    \SetKw{Return}{Return:}
    
    \Func MCTS($s_0$, $N$)
    
    \KwIn{original state $s_0$, search steps $N$}
    \KwOut{best leaf state $s*$}
    
	 new root node $v_0$ of the tree\; 
	 $v_0$.state $\leftarrow s_0$ \;
	 
	 \While{current search steps $< N$}{
	   $v_l \leftarrow $ tree\_policy($v_0$)\;
	   $\Delta \leftarrow $ default\_policy($v_l$.state)\;
	   backup($v_l, \Delta$)\;
	 } 
	 \Return{$s*$}
\end{algorithm}


\subsection{需要解决的问题与挑战}

测试一下数学公式中的字体大小。

\newcommand{\set}[1]{\left\{\,#1\,\right\}}
\newcommand{\card}[1]{\left|\,#1\,\right|}

Fall-Out指标计算公式如下:
\begin{equation*}
  \mbox{fallout} = \frac{\card{\set{\text{不相关文档}}\cap\set{\text{获取的文档}}}}{\card{\set{\text{不相关文档}}}}
\end{equation*}



\section{研究的应用背景}

\begin{figure}[htbp]
  \centering
  \includegraphics[width= 0.5\textwidth]{img/SchoolMark.jpg}\\
  \caption{测试插图}\label{fig:test1}
\end{figure}


阿卜·日·法拉兹曾经说过,学问是异常珍贵的东西,从任何源泉吸收都不可耻。带着这句话,我们还要更加慎重的审视这个问题\ref{fig:test1}:它发生了会如何,不发生又会如何。